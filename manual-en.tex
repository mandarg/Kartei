%
%    * ----------------------------------------------------------------
%    * "THE BEER-WARE LICENSE" (Revision 42/023):
%    * Ronny Bergmann <mail@rbergmann.info> wrote this file. As long as
%    * you retain this notice you can do whatever you want with this
%    * stuff. If we meet some day and you think this stuff is worth it,
%    * you can buy me a beer or a coffee in return.
%    * ----------------------------------------------------------------
%
%
% Kartei Source Code of the german manual
%
% Last Change: Kartei 1.9, 2012/01/04
%
\documentclass[a4paper,DIV=calc]{scrartcl}
\usepackage[utf8]{inputenc} %UTF8
\usepackage[OT1]{fontenc}
%\usepackage[scaled]{helvet}

% A bunch of packages mainly useful for mathematics and informatics
\usepackage{graphicx,hyperref}
\usepackage{enumerate,subfigure,listings}
\usepackage{pdfpages,color}
%\usepackage[pdftex,dvipsnames]{xcolor}
%\usepackage[light]{kpfonts}

% \lstloadlanguages{TeX}
\lstset{basicstyle={\rmfamily\footnotesize},numbers=left,numberstyle=\tiny\color{maincolor},numbersep=5pt, breaklines=true, captionpos={t},language={},frame=none,numbers=left,tabsize=3, showspaces=false, showtabs=false, columns=fixed}
\setlength{\parindent}{1em}
\setlength{\parskip}{0em}
\setlength{\marginparwidth}{3cm}
\colorlet{maincolor}{black!80}

\hypersetup{
   pdftitle={Flashcards in LaTeX}%
	  ,pdfauthor={Ronny Bergmann}%
	  ,pdfcreator={LaTeX, hyperref, KOMA-Script, TextMate},
   pdfkeywords={Cards, Karteikarten, TeX, Paket, Anleitung, Manual},
   pdfdisplaydoctitle,
   bookmarksnumbered=true,
   bookmarksopen=true,
   bookmarksopenlevel=1,
   plainpages=false,
   bookmarksnumbered,
   pdfborder={0 0 0}
   }

\addtokomafont{sectioning}{\rmfamily}
\setkomafont{descriptionlabel}{\rmfamily\bfseries}
\newcommand{\befehl}[1]{%
\marginpar{\footnotesize\textsf{#1}}%
}

\begin{document}
\title{Flashcards in \LaTeX\\{\large Version 1.9}}
\author{Ronny Bergmann\\\emph{mail@rbergmann.info}\footnote{English translation by Mandar Gokhale}}

\date{\today}
\maketitle
\section{Introduction}

Flashcards are helpful for learning, either in a small format for
vocabulary or in larger formats for more extensive topics. They can be
either viewed on the computer for revision, or printed. For this
purpose, a number of flashcards are placed on one page, while their
backs are printed on the next, so that if they are printed in duplex
format, the appropriate front and back pages for each card are printed
correctly on opposite sides of the paper. Both formats are based on
the same content, so \LaTeX\ can help to produce both of these
formats.


For testing, it is fine to put the file for the flashcards in the same
directory as the main class file. For more extensive projects, we
recommend putting should put the class files in a special folder:
(\lstinline!/texmf-dist/tex/latex/kartei)!, followed by running
\lstinline!sudo texhash!.

\subsection{Document Options}
For a set of flashcards \befehl{\textbackslash document-\\class\{kartei\}},
you can set the standard options for an article, for example the font size.
If unspecified, the font size is set to approximately \lstinline!10pt!.
  
\section{The Cards}
\befehl{aXpaper}
These are the following formats in which cards can be made:
\begin{description}
  \item[a5paper] Flashcards in DIN-A5-Format $(210$mm$\times 148$mm$)$\footnote{Thanks to M. Wolf for the idea}
  \item[a6paper] In DIN-A6-Karteikarten-Format $(148$mm$\times 105$mm$)$ the margins are slightly larger than in the other two formats. This format is the default if no format is specified.
  \item[a7paper] DIN-A7-Karteikarten $(105$mm$\times 74$mm$)$
  \item[a8paper] DIN-A8-Karteikarten $(74$mm$\times 52$mm$)$, from here onwards you should set the font
    smaller than 10pt.
  \item[a9paper] DIN-A9-Karteikarten $(52$mm$\times 37$mm$)$\footnote{Thanks H. Fritsch for the patch}  
\end{description}

\section{Technical Details}

\section{License}

\section{Changelog}

\end{document}
